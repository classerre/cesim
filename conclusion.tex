% Départ difficile
\paragraph{}
Après un départ tumultueux et un classement peu gratifiant, nous avons réussi
à remonter progressivement pour finalement arriver à la première place et y
rester. Ces faibles scores sont facilement expliquables par les nombreux
investissements à long terme que nous avons effectué pendant la première
année. L'accent que nous avons mis sur la recherche nous a handicapé au début
mais nous a propulsé au sommet du classement lorsque la couverture des
différentes zones s'est améliorée.

% Succès = tête froide + travail d'équipe
\paragraph{}
Notre progression constante est principalement dûe au fait que nous avons su
nous raisonner au sein du groupe, évitant ainsi des stratégies trop risquées
qui auraient pu nous mener à la catastrophe. Nous avons réussi à éviter les
conflits en faisant souvent des compromis entre les différents points de vue.
Plusieurs fois nous nous sommes rendus compte que si nous ne nous étions pas
écouter mutuellement, nous aurions fait de grosses erreurs.

% Fin difficile
\paragraph{}
Notre montée en flèche à mi-parcours s'est progressivement ralentie et au
final, d'autres équipes étaient bien parties pour remonter l'écart s'il y
avait eu plus de tours. Cette situation est dûe au fait que la simulation ne
permettait pas de continuer la recherche après la technologie 4. La limite de
tour nous a aussi décourager d'acheter de nouvelles usines, quelque chose que
nous aurions pu nous permettre de faire s'il restait plus d'années
d'exploitation. Nous sommes donc resté pendant plusieurs exercices avec des
fonds considérables sans trouver de réelles possibilités de le réinvestir.
Nous versions donc les plus grands dividendes possibles aux actionnaires, sans
pour autant réussir à écouler nos fonds. Or, il n'y a pas de situation plus
gênante pour une entreprise que de laisser dormir son argent.
