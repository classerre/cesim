Multinationale Américaine de téléphonie mobile, Archlinux 
est rapidement devenu un des leader de son domaine.
Devenu le premier constructeur mondial de téléphonie mobiles 
en 2013\footnote{classement établi par M. Eric ASTIEN},
ArchMobile s'est illustré sur ces 5 dernières années par
la confiance qu'il a su établir au sein d'une clientelle 
de qualité. Malgré un essai peu concluant dans le low-cost,
leurs produits atteignent désormais une qualité, un apport
technologique et une fiabilité inégalés sur l'ensemble du
marché.



\paragraph{\'Etymologie}~\\
Le nom ``ArchMobile'' provient de la distribution ``ArchLinux''
intégrée à  chaque téléphone. L'hypothèse la plus plausible
est que les fondateurs de la société se soient inspirés du premier
nom qu'ils aient entendu.

\paragraph{Produits}~\\
ArchMobile s'est imposé progressivement sur le marché avec 3 technologies:
\begin{itemize}
\item ArchT1 : incluant la Tech1, sorti rapidement sur le marché et qui a assuré 
  des revenus réguliers sur les premières années de la compagnie.
\item ArchT3 : produit phare, sorti après une période de disette 
  économique et étant le fruit de gros efforts de la société à long terme
\item ArchT4 : le dernier produit high-tech de ArchMobile qui assoit
  sa différenciation par le haut sur le marché.
\end{itemize}

\paragraph{Valeurs}
\subparagraph{Ecologie}
Depuis toujours, Archmobile concilie écologie, responsabilité et
design. Tous nos composant sortent d'usine certifiée MAX HAVELAAR, et
tout nos employés ont a coeur la conservation de notre planète.
\subparagraph{Entraide}
Archmobile s'engage également régulièrement au côté d'association
caritative, et à créer sa propre fondation ``les orphelins de
l'Arche'' qui aide des enfants victimes de la guerre a retrouver un
foyer.
\subparagraph{Art}
Archmobile aime les arts et le prouve en organisant tout les ans le
festival du cinémé d'art et d'essai de Guingamp, ainsi que le festival
de danse colombienne ancienne de Montauban.
\subparagraph{Science}
Depuis deux ans, Archmobile soutient l'association d'aide au mythomane
(l'AAM) en lui permettant d'approcher des homme politiques de tout
bords.
