\paragraph{Première année sous notre direction:}~\\
Notre strategie était basée sur l'évolution de la couverture des
antennes dans les diverses zones géographiques. Nous voulions jouer
sur la montée en puissance de la Tech3 en Asie puis aux États-unis,
tout en gardant une base de vente régulière avec la Tech1. Nous
avions conscience que d'autres équipes allaient pouvoir attaquer 
rapidement sur la Tech2 et pouvoir, en arrivant avant les autres,
se faire des marges plus importantes. Nous avons donc choisis de 
laisser cette technologie pour les autres.\\

Pour cela nous avons d'emblés embauché massivement des ingénieurs, 
en augmentant de 600 à 1500 dollars le budget de la formation par 
ingénieur, et en augmentant les salaires des ingénieurs a 4800
dollars par mois. Le prix des licences étant de plus en plus chers
nous voulions avoir notre propre force de travail.
Les forces de travail ont été répartie sur les options de la Tech1
et d'emblée sur la Tech3.\\

Nous voulions rapidement faire du chiffre, même peu. Nous avons donc 
aussi investi dans 10 usines en Chine, où l'on se doutez que les prix 
de gestion et taxes seraient moindres. Nous découvrions pourtant
ces paramètres que quelques tours plus tard. Nous avons aussi pour cela
acheter 5 licences sur la Tech1 que l'on savait que l'on pourrait 
conserver longtemps sur le marché asiatique d'après sa description.\\
 
Pour avoir des parts de marché importante (suites aux tours d'entraînement
nous estimions avoir 18\% des parts de marché sur la Tech1), nous
avons enfin fait beaucoup de publicité : (un maximum) $100 000 kUSD$.
Nous avons enfin veillé avoir avoir des marges satisfaisantes sur chaque
marché de manière à rentré dans nos investissements, surtout en Amérique
et en Europe. La conquête de l'asie viendrait avec les usines l'année 
prochaine.\\

Nous visions sur ces quelques critères de base de faire un résultat
correcte. En sachant que notre bilan prendrait son temps dans 3-4 tours
avec l'arrivée de la Tech3 et des usines en bon fonctionnement.


\paragraph{La crise des télécoms:}~\\

Ce choix, bien qu'il se soit révélé payant, à été compliqué à gérer
notamment durant la troisième année, première année de la dure crise
qui a touché le secteur des télécoms. En effet les perspectives
étaient plutôt sombres, nos concurrents faisaient de bon chiffres sur
la tech 2, nos ventes de télephone tech 1 commençait à s'essouffler et
le marché n'était pas prêt à acceuillir la tech 3. Nous avons donc
décidé de prendre un minimum de risque sur ce tour, en diminuant notre
production et en tatant le terrain en Asie avec la tech 3. Nous avons
également émis un certain nombre d'action pour pouvoir racheter nos
dettes et diminuer le poids des interets sur nos résultats.

\paragraph{La reprise:}~\\

La quatrième année à été un tournant pour notre compagnie. Alors que nos
concurrents ont accumulé les pertes, nous avons eu une bonne année, en
prenant 3\% de part de marché global et en ayant des pertes très
mesurées. Notre compagnie representait près de 24\% du marché
asiatique, très loin devant la plupart de nos concurrents.

\paragraph{L'envolée:}~\\

Notre compagnie a commencée a réelement prendre son envol durant la
cinquième année. nos capacité de production importante ainsi que notre
R\&D efficace nous a permis de sortir de meilleurs téléphones
(introducton de la tech 4 en europe) et d'améliorer nos marges. On
observe aussi que notre chiffre d'affaire est superieur de 30\% a
celui de la majorité de nos concurrents. Devant la montée en puissance
de la capacité de production de nos concurrents, nous avons fait le
choix de ne pas augmenter notre nombre d'usine, en profitant de notre
bonne santé financière pour viser un repositionnent de notre gamme de
téléphone vers le haut de gamme. Ce sera notamment la dernière année
ou nous produirons de la tech 1.

\paragraph{L'age d'or:}~\\

La sixième année fut une année faste pour Archmobile. Des résultats
records et de solides bénéfices ont consacrée Archmobile comme la
première entreprise du secteur. Nous avons profités de nos liquidités
importantes pour commencer a verser des dividendes a nos actionnaires,
et à racheter des actions. Décisions est également prise de réduire la
voilure au niveau du personnel en commençant à diminuer le salaires de
nos ingénieurs, afin d'améliorer nos marges.

\paragraph{Les premières turbulances:}~\\

La septième année a été difficile pour nous, après une excellente
sixième année, nous étions ratrappé par nos concurrent, qui avait
cassé les prix et pris nos part de marché au point que nous n'avions
jamais eu autant de stock a la fin d'un exercice (plus de quatre
millions de mobile). Notre chiffre d'affaire était également en baisse.
Malgré tout,nos bénéfices restaient confortable.

\paragraph{La reconquête:}~\\
 
Nous avons décidés sur le huitième exercice de diminuer nos marges et
de vider nos stocks.  Ce fut en partie une erreur, car si nous avons
effectivements regagnés des parts de marchés, nos profit et notre
rentabilité financière ont baissée. Non seulement nous n'avons pas
pris de part de marché a nos adversaires du fait de capacité de
production insuffisante, mais certain de nos concurrents ont des chiffres d'affaires 
comparable au notre, et même des benefices superieurs. 

