Notre strategie était basée sur l'évolution de l'implémentation des
antennes dans les diverses zones géographiques. Nous voulions jouer
sur la montée en puissance de la tech 3 en asie puis aux états-unis,
tout en gardant une base de vente régulière avec la tech 1. Pour cela
nous avons d'emblé fait des investissements lourds dès la première
année, en achetant dix nouvelles usines en Asie, en embauchant
massivement des ingénieurs, en augmentant de 600 à 1500 dollars le
budget de la formation par ingénieur, et en augmentant les salaires
des ingénieurs a 5000 dollars par mois.


Ce choix, bien qu'il se soit révélé payant, à été compliqué à gérer
notamment durant la troisième année, première année de la dure crise
qui a touché le secteur des télécoms. En effet les perspectives
étaient plutôt sombres, nos concurrents faisaient de bon chiffres sur
la tech 2, nos ventes de télephone tech 1 commençait à s'essouffler
et le marché n'était pas prêt à acceuillir la tech 3. Nous avons donc
décidé de prendre un minimum de risque sur ce tour, en diminuant notre
production et en tatant le terrain en Asie avec la tech 3. Nous avons également émis un certain nombre d'action pour pouvoir racheter nos dettes et diminuer le poids des interets sur nos résultats.


La quatrième année à été tournant pour notre compagnie. Alors que nos
concurrents ont accumulé les pertes, nous avons eu une bonne année, en
prenant 3\% de part de marché global et en ayant des pertes très
mesurées. Notre compagnie representait près de 24\% du marché
asiatique, très loin devant la plupart de nos concurrents.


Notre compagnie a commencée a réelement prendre son envol au cinquième tour. nos capacité de production importante ainsi que notre R\&D efficace nous a permis de sortir de meilleurs téléphones (introducton de la tech 4 en europe) et d'améliorer nos marges. On observe aussi que notre chiffre d'affaire est superieur de 30\% a celui de la majorité de nos concurrents. Devant la montée en puissance de la capacité de production de nos concurrents, nous avons fait le choix de ne pas augmenter notre nombre d'usine, en profitant de notre bonne santé financière pour viser un repositionnent de notre gamme de téléphone vers le haut de gamme. Ce sera notamment la dernière année ou nous avons produit la tech 1.

 

